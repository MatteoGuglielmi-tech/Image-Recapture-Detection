\paragraph{Abstract}
Nowadays, with the availability of sofisticated digital cameras, images can be acquire with high-quality also from LCD monitor screens quite easily. In light of this, an hypothetical attacker may want to reacpture a forged image in order to hide imperfections and to increase its authenticity.\\
The underlying idea of this work is to develop an algorithm, following a piece of the paper cited in the bibliography \cite{paper}, which is able to recognize whether a specific image has been recaptured or not. \\
As quoted in the reference article, the authors address that '[...] aliasing and blurriness are the least scene dependent features'; since aliasing can be ereased by setting the parameters of the recapturing system to predetermined values, in this work we focus on the bluriness feature in order to perform the final recognition. More precisely, a set of learned edge bluriness are used to perform a classification of each input image.\\
Every time a scene is captured with a digital camera, a certain amount of blur is introduced into the picture by the image acquisition system. This blur can be characterized by the so called Point Spread Function of the capture device. Unfortunately, the PSF of a system is not easily achievable and the Line Spread Function (LSF) is used instead. 'By definition, a line spread function is a $1-$D function
corresponding to the first derivative of the edge spread function (ESF)' \cite{bluriness}\\
Another important component are the two over-complete dictionaries built by means of the K-SVD algorithm, one for both recaptured and single captured images in the available dataset \cite{database}. Each dictionary is trained to provide an optimal sparse representation of the Line Spread Profile function extracted from the example images. The purpose of the two dictionaries is to describe how well wach dictionary fits the line spread profile matrix ($Q_i$) of a query image. In poor words, an approximation error is computed as a measure of likeliness between an unknown query image and $D_{SC}$ and $D_{RC}$.\\
These dictionary, which will be referred to as $D_{SC}$ and $D_{RC}$ for the dictionary trained with single captured and recaptured images respectively, have a key role the approximation error which is one of the two features used.\\ The other feature used, $\lambda_{avg}$, refers to the average of Line Spread Profile widths computed as follows : the value of $\lambda$ is defined as the minimum distance of the ESF such that at least $95\%$ of the spectral energy of the spread function falls within it. \\
Eventually, after having computed $\lambda_{avg}$ and $E_d$ for every image of the training set, a support vector machine classifier (SVC) is trained.\\
Whenever an unknown query image has to be classified, the aforesaid features has to be computed and then feed the already built SVM with the respective values.